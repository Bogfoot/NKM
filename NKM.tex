%! TEX program = pdftex

\documentclass{article}
\usepackage{graphicx}
\usepackage[croatian]{babel}
\usepackage{braket}
\usepackage[utf8]{inputenc} % allow utf-8 input
\usepackage[T1]{fontenc}    % use 8-bit T1 fonts
\usepackage{hyperref}       % hyperlinks
\usepackage{url}            % simple URL typesetting
\usepackage{booktabs}       % professional-quality tables
\usepackage{amsfonts}       % blackboard math symbols
\usepackage{nicefrac}       % compact symbols for 1/2, etc.
\usepackage{amsmath}
\usepackage{microtype}      % microtypography
\usepackage{geometry}
\usepackage{subfigure}
%\usepackage{showframe}%

\newgeometry{vmargin={20mm}, hmargin={15mm,15mm}}   % set the margins
\renewcommand{\figurename}{Slika}
\newtheorem{theorem}{Teorem}[section]
\newtheorem{proof}{Dokaz}[section]
\numberwithin{equation}{section}

% \newenvironment{thm}[1]
%     {\begin{center}
%     #1\\[1ex]
%     \begin{tabular}{|p{0.9\textwidth}|}
%     \hline\\
%     }
%     { 
%     \\\\\hline
%     \end{tabular} 
%     \end{center}
%     }


% \renewcommand*\contentsname{Sadržaj}
 
\title{Skripta: Napredna Kvantna Mehanika}


\author{Adrian Udovičić}

\begin{document}
\maketitle
\pagenumbering{roman}
\section*{\centering Uvod}

Svrha ove skripte je upoznati studente različitih smjerova s kolegijem napredne kvantne mehanike.

\vfill

\begin{center}
\tableofcontents
\end{center}
\vfill
\newpage
%% početak prvog poglavlja
\pagenumbering{arabic}

\section{Ket i Bra notacija}
\subsection{Vektori baze i matrična reprezentacija}
Kvantno stanje opisujemo sa $\ket{\psi}$ koji razapinju Hilbertov prostor s definiranim skalarnim produktom i svojstvima:
\begin{itemize}
	\item $\ket{\alpha} + \ket{\beta} = \bra{\gamma}$,
	\item c $\cdot \ket{\alpha} = \ket{c \cdot \alpha}$,
	\item Postoji operator $\hat{A}$ t.d. $\ket{\beta}$.
\end{itemize}
Dimenzija prostora određenja je problemom, npr. za 1 elektron imamo dimenzije momenta,
angularnong momenta i spina $\rightarrow$ 3 + 3 + 2 = 8D. Baza  $\left\{\ket{\alpha}\right\}$, gdje ket $\ket{\alpha}$ sadrži sve informacije o sistemu, ali do informacije dolazimo skalarni produkt:
$\bra{\beta}\ket{\alpha}$, gdje je $\bra{\beta}$ iz dualnog prostora ket prostora.
Korespondencija: 
\begin{equation}
	C_1\ket{\alpha} + C_2 \ket{\beta} \Leftrightarrow C_1^*\bra{\alpha} C_2^*\bra{\beta}
\end{equation}
Svojstva skalarnog produkta:
\begin{itemize}
	\item  $\bra{\beta}\ket{\alpha} = \bra{\alpha}\ket{\beta}^*$,
	\item Pozitivna definitnost $\Rightarrow$ $\bra{\alpha}\ket{\alpha} > 0$,
	\item Ortogonalnost $\Rightarrow$ $\bra{\alpha}\ket{\beta} = 0 =\bra{\beta}\ket{\alpha}$,
	\item Normalizacija $\Rightarrow$ $\ket{\hat{\alpha}} = \frac{\ket{\alpha}}{\sqrt{\bra{\alpha}\ket{\alpha}}} \Rightarrow \braket{\alpha \alpha} = 1$. 
\end{itemize}
Svaki vektor možemo razviti po vektorima baze:
\begin{equation}
\ket{\alpha} = \sum_{n=1}^{N}c_n\ket{\alpha}; c_n = \braket{\alpha_n|\alpha}
\end{equation}

\subsubsection{Operatori}
Operatori predstavljaju observable, tj. pomoću operatora pridružujemo fizikalne veličine opažanom sustavu. Npr. Operator položaja $\hat{x}$
koristi se za određivanje položaja sustav, operator momenta $\hat{p}$ za određivanje momenta, itd.
Dijelovanje operatora na neko stanje može promijeniti to stanje:
\begin{equation}
	\hat{A}\ket{\alpha} = \ket{\beta}
\end{equation}
tj.
\begin{equation}
\hat{A}\ket{\alpha} = a \ket{a},
\end{equation}
gdje je $\ket{a}$ svojstveni vektor (svj. v.) operatora $\hat{A}$, a $a$ svojstvena vrijednost (svj. vrij.).\\

Svojstva operatora:
\begin{itemize}
	\item Zbrajanje
	\begin{itemize}
	\item Komutativnost: $\hat{A} + \hat{B} = \hat{B} + \hat{A}$,
	\item Asocijativnost: $\hat{A} + \left(\hat{B} + \hat{C}\right) = \left(\hat{A} + \hat{B} \right) \hat{C}$,
	\item Linearnost: $\hat{A}\left(a\ket{a}+b\ket{b}\right) = a\hat{A}\ket{a} + b\hat{A}\ket{b}$,
	\item Hermetičnost: $\hat{A}\ket{a} = \bra{a}\hat{A}^*$
	\end{itemize}
\item Množenje
	\begin{itemize}
		\item (Anti-)Komutativnost: $\hat{A}\hat{B} \ne \hat{B} \hat{A}$,
		\item Asocijativnost: $\hat{A}\left(\hat{B}\hat{C}\right) = \left(\hat{A}\hat{B}\right)\hat{C}$,\\
			tj. $\hat{A}\left(\hat{B}\ket{a}\right)=\hat{A}\hat{B}\ket{a}$,
		\item Hermetičnost: $\left(\hat{A}\hat{B}\right)^{\dagger} = \hat{B}^{\dagger}\hat{A}^{\dagger}$,\\
			tj. $\hat{A}\hat{B} \ket{a} = \bra{a} \hat{B}^{\dagger} \hat{A}^{\dagger}$.
	\end{itemize}

\end{itemize}
\subsubsection{Komutatori}
\begin{equation}
	\left[\hat{A},\hat{B}\right] = \hat{A} \hat{B} - \hat{B}\hat{A}
\end{equation}
U klasičnoj mehanici to s ubili generatori koordinata i impulsa. Glavna značajka prelaska iz klasične u kvantnu mehaniku je $\left[p,q\right]\ne0=i\hbar$.
\subsubsection{Vanjski produkt}
\begin{equation}
	\ket{\beta}\bra{\alpha}
\end{equation}
je operator sa sljedećim svojstvima:
\begin{itemize}
	\item $\left(\ket{\beta}\bra{\alpha}\right)\ket{\gamma} = \ket{\beta}\braket{\alpha|\gamma} = c_{\alpha\gamma}\ket{\beta}$,
	\item $\hat{X} = \ket{\beta}\bra{\alpha} \rightarrow \hat{X}^{\dagger} = \ket{\alpha}\bra{\beta}$,
	\item $\left(\bra{\beta}\right)\hat{X}\ket{\alpha} = \braket{\beta|\hat{X}|\alpha} = \left(\braket{\alpha|\hat{X}^{\dagger}|\beta}\right)^{\dagger}$,
	\item $\braket{\beta|\hat{X}|\alpha} = \braket{\alpha|\hat{X}|\beta}^{\dagger}$.
\end{itemize}
\subsubsection{Hermitski operatori}
 $\hat{A} = \hat{A}^{\dagger}$ je jednadžba hermetičnosti. Ako jednakost vrijedi operator $\hat{A}$ je hermetičan.
 \begin{theorem}
	 Svojstvene vrijednosti hermitskog operatora su realne, a vektori su međusobno ortogonalni.
 \end{theorem}
\begin{proof}
	\begin{equation}
		\begin{aligned}
			\hat{A}\ket{n}=& a_n\ket{n}\\
			\bra{n}\hat{A}^{\dagger} =& \bra{n}a_n^*\\
			\braket{n|\hat{A}|m} =& \braket{n|m}a_n^*\\
			\a_m\braket{n|m} =& a_n^* \braket{n|m} \Rightarrow(a_n^*-a_m)\braket{n|m} = 0
		\end{aligned}
	\end{equation}
\end{proof}
Imamo 2 slučaja: 
\begin{itemize}
	\item $n = m$
	\item $n \ne m$
\end{itemize}
Za $n = m  \Rightarrow a_n^* = a_n \Rightarrow \braket{n|m}>0$ (Nisu ortogonalni) t.d.:
\begin{itemize}
	\item   $\forall\ket{m} \ne \ket{n} \Rightarrow a_m\ne \a_n \Rightarrow a_n - a_m \ne 0 \Rightarrow \braket{n|m} = 0$
	\item   $\forall\ket{m} \ne \ket{n} \Rightarrow a_m\ne \a_n \rightarrow$ 
\end{itemize}
različite funkcije s istim stanjima\\
$\righarrow$ Koliko različitih svojstvenih funkcija može imati istu svojstvenu vrijednost?\\
Za $k \ne \ket{m} \exists k $ istih svojstvenih vrijednosti za k različitih svojstvenih vektora $\Rightarrow$ \underline{k puta degenerirane svojstvene vrijednosti}\\
$\Rightarrow$ Kako je $a_n - a_m = 0 \Rightarrow \braket{n|m} =$? U prostoru možemo izabrati vektore koji su međusobno ortogonalni (Gram-Schmidtov postupak ortogonalizacije), te time dobivamo
$\braket{n|m} = \delta_{n,m}$ što je relacija ortonormiranosti.
Vlastite funkcije (time i vektori) hermitskog operatora čine potpun skup, sve druge vektore možemo napisati kao linearnu kombinaciju pa su to vektori baze $\left\{\ket{n}\right\}$
\begin{equation}
	\ket{\alpha} = \sum_n c_n \ket{n} \Rightarrow \braket{m|\alpha} = \sum_n c_n\braket{m|n} = c_m
	\label{EQPotpunosti} % Ova doli je prava relacija potpunosti
\end{equation}

\begin{equation}
	\ket{\alpha} = \sum_n c_n \ket{n} = \ket{n}\braket{n|\alpha} \Rightarrow \sum_n \ket{n}\bra{n} = 1.
	\label{EQPotpuna}
\end{equation}

Zadnji dio jednadžbe \ref{EQPotpuna} je relacija potpunosti.\\
Vjerojatnost:
\begin{equation}
	\begin{aligned}
		\braket{\alpha|\alpha} =& \braket{\alpha|\sum_n|n}\braket{n|\alpha}
		&= \sum_n \left|\braket{n|\alpha}\right|^2 = \sum_n \left|c_n\right|^2=1
	\end{aligned}
\end{equation}
Gornji izvod nam govori da ćemo uvjek, npr.naći česticu u nekom stanju. Uvjek ćemo "nešto" izmjeriti $\Rightarrow$ \underline{normalizacija}
(Jer $\sum_n\left|c_n\right|^2 = c_1c_1^* + c_2c_2^*...$).
\subsubsection{Matrična reprezentacija}
Matrična reprezentacija operatora dana je s:
\begin{equation}
	\hat{X} = \sum_m\sum_n \ket{m}\braket{m|\hat{X}|n}\bra{n}
\end{equation}
gdje su nam 'm'-ovi redovi, a 'n'-ovi stupci, tj.
\begin{equation}
	\hat{X} = 
\begin{bmatrix}
	\braket{1|\hat{X}|1} & \braket{1|hat{X}|2} & \braket{1|\hat{X}|N}\\
	... & ... & ...\\
	\braket{N|\hat{X}|1}& ...& \braket{N|\hat{X}|N}
\end{bmatrix}
\end{equation}
Ako je $\hat{X}$ hermitski operator ($\hat{X} = \hat{X}^{\dagger}$), onda :
\begin{equation}
	\begin{aligned}
		\braket{m|\hat{X}|n} &= \braket{n|\hat{^{\dagger}|m}}^{\dagger}\\
		\braket{m|\hat{X}|n} &= \braket{n|\{X}|m^{\dagger}\\
			\Rightarrow \hat{X_{m,n}} &= \hat{X_{m,n}}^{\dagger}
	\end{aligned}
\end{equation}
Imamo više reprezentacija\footnote{U smislu da imamo više stvari za "matrično reprezentirati".}:
\begin{itemize}
	\item ket: $\ket{\alpha} = \sum_n\ket{n}\braket{n|\alpha} =
				\begin{bmatrix} \braket{1|\alpha}\\ ...\\ \braket{n|\alpha} \end{bmatrix} 
				= \begin{bmatrix} c_1\\ ...\\ c_n  \end{bmatrix}$,\\
				gdje je $\braket{n|\alpha} = c_n$,
	\item bra: $\bra{\alpha} = \sum_n \braket{\alpha|n}\bra{n} = \begin{bmatrix} c_1^* & ... & c_n^* \end{bmatrix}$,
	\item skalarni produkt: $\braket{\beta|\alpha} = \sum_n \braket{\beta|n}\braket{n|\alpha}=
				\begin{bmatrix} c_1^* & ... & c_n^* \end{bmatrix} \begin{bmatrix} c_1\\ ...\\ c_n  \end{bmatrix}$.
\end{itemize}
\subsection{Mjerenja, tj. opservable i relacija neodređenosti}
\subsubsection{Mjerenja (opservable)}
Sustav se prije mjerenja nalazi u tzv. "superpoziciji" stanja (\ref{EQPotpunosti}). Tijekom mjerenja
sustav pređe iz te neodređene linearne kombinacije stanja u neko od svojih svostvenih stanja.
Sustav koji se već nalazi u svojstvenom stanju prije mjerenja, ostat će u njemu i tijekom mjerenja\footnote{Treba uzeti u obzir da sustav ostane
u nekom stanju neko konačno vrijeme, ne $\infty$ dugo vremena.}.
\begin{theorem}
	Postulat: $\ket{\alpha} \xrightarrow{mjerenje} \ket{\alpha_n}.$
\end{theorem}
Opet, vjerojatnost da izmjerimo stanje $\ket{\alpha_n}$ dana je s: 
\begin{equation}
	W_n = \left|c_n\right|^2 = \left|\braket{\alpha_n|\alpha}\right|^2.
\end{equation}
Također,
\begin{equation}
	\sum_n \left|c_n\right|^2 = \sum_n c_n^*c_n = \sum_n \braket{\alpha_n|\alpha}^*\braket{\alpha_n|\alpha} = \sum_n \braket{\alpha|\alpha_n}\braket{\alpha_n|\alpha}
	=\sum_n \braket{\alpha|\alpha}= 1
\end{equation}
Očekivana vrijednost operatora dana je na sljedeći način:
\begin{equation}
	\begin{aligned}
		\braket{\hat{A}} &= \braket{\alpha|\hat{A}|\alpha}\\
										 &= \sum_{m,n} \braket{\alpha|\alpha_m}\braket{\alpha_m|\hat{A}|\alpha_n}\braket{\alpha_n|\alpha}\\
										 &= \sum_{m,n} c_m^*c_n \braket{\alpha_m|\hat{A}|\alpha_n}\\
										 &= \sum_{m,n} c_m^*c_n a_n \braket{\alpha_m|\alpha_n}\\
										 &= \sum_{m,n} c_m^*c_n a_n \delta_{m,n}\\
										 &= \sum_{n} c_n^*c_n a_n\\
		\braket{\hat{A}} &= \sum_n a_n \left|c_n\right|^2  
	\end{aligned}
\end{equation}

\subsubsection{Kompatibilne i nekompatibilne varijable\footnote{U daljnjem tekstu neću koristiti notaciju 'kapica' za operatore jer mi se ne da.}}
Pri vršenju mjerenja u kvantoj mehanici bitan je redosljed mjerenja opservabli. Ako kvantno mehaničkom sustavu idemo mjeriti
položaj te moment, a zatim mjerimo nad istim sustavom (ignorirajući činjenicu da smo upravo promjenili sustavu,
pretpostavljamo da su oba sustava identična) moment te položaj dobit ćemo različite rezultate i takve opservable nazivamo nekompatibilnim,
tj., nekompatibilnim varijablama.\\
Kod kompatibilnih varijabli nam redosljed mjerenja nije bitan.
\paragraph{Kompatibilne varijable} 

\begin{equation}
	\left[A,B\right] = 0
\end{equation}
Ako A i B imaju iste svojstvene vrijdnosti ($\exists$ degeneracija stanja) $\Rightarrow$ kao bazu za operator 
A možemo koristiti svojstvene vektore operatora B i obratno, tj. B možemo razviti preko svojstvenih vektora operatora A i obratno.
\begin{theorem}
	Neka su A i B kompatibilne varijable i neka je A nedegenriran $\Rightarrow$ B je dijagonalan ($\braket{\alpha_m|B|\alpha_n} = \delta_{m,n}$)
\end{theorem}
\begin{proof}
	\begin{equation}
		\begin{aligned}
			\braket{\alpha_m|\left[A,B\right]|\alpha_n} &= \braket{\alpha_m|AB-BA|\alpha_m}\\
																									&=\braket{\alpha_m|AB|\alpha_n} - \braket{\alpha_m|BA|\alpha_n}\\
																									&= a_m \braket{\alpha_m|B|\alpha_n} - a_n\braket{\alpha_m|B|\alpha_n}\\
																									&=\left(a_m - a_n \right) \braket{\alpha_m|B|\alpha_n} = 0
		\end{aligned}
	\end{equation}
	U zadnjoj liniji možemo izjednačiti s nulom jer očito ako su A i B kompatibilne varijable, u prvom redu ćemo imati 0 = $\braket{\alpha_m|AB-BA|\alpha_m}$
	t.d. nam ili zagrada u zadnje redu mora biti 0, ili braket. Pošto vrijedi $a_m \ne a_n$, onda mora vrijediti teorem.
\end{proof}



\paragraph{Nekompatibilne varijable}

\subsubsection{Relacije neodređenosti}

\subsection{Promjena baze}

\subsection{Položaj, moment i translacija}

\subsection{Valna funkcija u prostoru položaja i momenta}




\newpage
\section{Kvantna dinamika}

\newpage
\section{Teorija angularnog momenta}

\newpage
\section{Simetrije Kvantne mehanike}

\newpage
\section{Aproksimacijske metode}

\newpage
\section{Teorija raspršenja}
\end{document}
