%! TEX program = pdftex

\documentclass{article}
\usepackage{graphicx}
\usepackage[croatian, english]{babel}
\usepackage{braket}
\usepackage[utf8]{inputenc} % allow utf-8 input
\usepackage[T1]{fontenc}    % use 8-bit T1 fonts
\usepackage{hyperref}       % hyperlinks
\usepackage{url}            % simple URL typesetting
\usepackage{booktabs}       % professional-quality tables
\usepackage{amsfonts}       % blackboard math symbols
\usepackage{nicefrac}       % compact symbols for 1/2, etc.
\usepackage{amsmath}
\usepackage{microtype}      % microtypography
\usepackage{geometry}
\usepackage{subfigure}
%\usepackage{showframe}%

\newgeometry{vmargin={20mm}, hmargin={15mm,15mm}}   % set the margins
\renewcommand{\figurename}{Slika}
   \newtheorem{theorem}{Teorem}
\numberwithin{equation}{section}

\renewcommand*\contentsname{Sadržaj}

\title{Skripta: Napredna Kvantna Mehanika}


\author{Adrian Udovičić}

\begin{document}
\maketitle
\pagenumbering{roman}
\section*{\centering Uvod}

Svrha ove skripte je upoznati studente različitih smjerova s kolegijem napredne kvantne mehanike.

\vfill

\begin{center}
\tableofcontents
\end{center}
\vfill
\newpage
%% početak prvog poglavlja
\pagenumbering{arabic}

\section{Ket i Bra notacija}
\subsection{Vektori baze i matrična reprezentacija}
Kvantno stanje opisujemo sa $\ket{\psi}$ koji razapinju Hilbertov prostor s definiranim skalarnim produktom i svojstvima:
\begin{itemize}
	\item $\ket{\alpha} + \ket{\beta} = \bra{\gamma}$,
	\item c $\cdot \ket{\alpha} = \ket{c \cdot \alpha}$,
	\item Postoji operator $\hat{A}$ t.d. $\ket{\beta}$.
\end{itemize}
Dimenzija prostora određenja je problemom, npr. za 1 elektron imamo dimenzije momenta,
angularnong momenta i spina $\rightarrow$ 3 + 3 + 2 = 8D. Baza  $\left\{\ket{\alpha}\right\}$, gdje ket $\ket{\alpha}$ sadrži sve informacije o sistemu, ali do informacije dolazimo skalarni produkt:
$\bra{\beta}\ket{\alpha}$, gdje je $\bra{\beta}$ iz dualnog prostora ket prostora.
Korespondencija: 
\begin{equation}
	C_1\ket{\alpha} + C_2 \ket{\beta} \Leftrightarrow C_1^*\bra{\alpha} C_2^*\bra{\beta}
\end{equation}
Svojstva skalarnog produkta:
\begin{itemize}
	\item  $\bra{\beta}\ket{\alpha} = \bra{\alpha}\ket{\beta}^*$,
	\item Pozitivna definitnost $\Rightarrow$ $\bra{\alpha}\ket{\alpha} > 0$,
	\item Ortogonalnost $\Rightarrow$ $\bra{\alpha}\ket{\beta} = 0 =\bra{\beta}\ket{\alpha}$,
	\item Normalizacija $\Rightarrow$ $\ket{\hat{\alpha}} = \frac{\ket{\alpha}}{\sqrt{\bra{\alpha}\ket{\alpha}}} \Rightarrow \braket{\alpha \alpha} = 1$. 
\end{itemize}
Svaki vektor možemo razviti po vektorima baze:
\begin{equation}
\ket{\alpha} = \sum_{n=1}^{N}c_n\ket{\alpha}; c_n = \braket{\alpha_n|\alpha}
\end{equation}

\subsubsection{Operatori}
Operatori predstavljaju observable, tj. pomoću operatora pridružujemo fizikalne veličine opažanom sustavu. Npr. Operator položaja $\hat{x}$
koristi se za određivanje položaja sustav, operator momenta $\hat{p}$ za određivanje momenta, itd.
Dijelovanje operatora na neko stanje može promijeniti to stanje:
\begin{equation}
	\hat{A}\ket{\alpha} = \ket{\beta}
\end{equation}
tj.
\begin{equation}
\hat{A}\ket{\alpha} = a \ket{a},
\end{equation}
gdje je $\ket{a}$ svojstveni vektor (svj. v.) operatora $\hat{A}$, a $a$ svojstvena vrijednost (svj. vrij.).\\

Svojstva operatora:
\begin{itemize}
	\item Zbrajanje
	\begin{itemize}
	\item Komutativnost: $\hat{A} + \hat{B} = \hat{B} + \hat{A}$,
	\item Asocijativnost: $\hat{A} + \left(\hat{B} + \hat{C}\right) = \left(\hat{A} + \hat{B} \right) \hat{C}$,
	\item Linearnost: $\hat{A}\left(a\ket{a}+b\ket{b}\right) = a\hat{A}\ket{a} + b\hat{A}\ket{b}$,
	\item Hermetičnost: $\hat{A}\ket{a} = \bra{a}\hat{A}^*$
	\end{itemize}
\item Množenje
	\begin{itemize}
		\item Komutativnost: $\hat{A}\hat{B} \ne \hat{B} \hat{A}$,
		\item 
	\end{itemize}

\end{itemize}


\subsection{Mjerenja, opservable i relacija neodređenosti}

\subsection{Promjena baze}

\subsection{Položaj, moment i translacija}

\subsection{Valna funkcija u prostoru položaja i momenta}




\section{Kvantna dinamika}

\section{Teorija angularnog momenta}

\section{Simetrije Kvantne mehanike}

\section{Aproksimacijske metode}

\section{Teorija raspršenja}
\end{document}
